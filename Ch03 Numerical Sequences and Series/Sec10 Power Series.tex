\documentclass[../poma-notes.tex]{subfiles}

\begin{document}

\subsection*{Power Series}

\begin{definition}
  Given a sequence $\{c_n\}$ of complex numbers, the series
  \begin{equation}
    \sum_{n=0}^{\infty} c_n z^n
  \end{equation}
  is called a \textit{power series}. The numbers $c_n$ are called the \textit{coefficients} of the series;
  $z$ is a complex number.
\end{definition}

通常而言,级数的收敛或发散取决于 $z$ 的选择。更确切的来说,每个幂级数都有一个关联圆,称为\textbf{收敛圆},如果 $z$ 在圆内,
幂级数就收敛,如果在圆外则发散(这里把平面看做半径无限大的圆的内部,把一点看作是半径为零的圆)。级数在收敛圆上的性质不能简单的叙述。

\anote 对于复数序列 $\{c_n\}$,级数 $\sum_{n=0}^{\infty}c_n z^n$ 叫做\textbf{幂级数},$c_n$ 叫做这个级数的\textbf{系数}。

\begin{anote}
  单变量的幂级数是一个拥有下列形式的无限级数:
  \[
    \sum_{n=0}^{\infty} a_n(x-c)^n = a_0 + a_1(x - c) + a_2(x - c)^2 + \cdots
  \]
  其中 $a_n$ 代表第 $n$ 项的系数,$c$ 为一个常数。
\end{anote}

\begin{theorem}
  Given the power series $\Sigma c_n z^n$, put
  \[
    \alpha = \limsup_{n\to\infty} \sqrt[n]{|c_n|}, \qquad R = \frac{1}{\alpha}.
  \]
  (If $\alpha=0, R=+\infty$; if $\alpha=+\infty, R=0$.) Then $\Sigma c_n z^n$ converges if $|z| < R$, and
  diverges if $|z| > R$.
\end{theorem}

\begin{proof}
  令 $a_n = c_n z^n$,并且应用根值审敛法:
  \[
    \limsup_{n\to\infty}\sqrt[n]{|a_n|} = |z|\limsup_{n\to\infty}\sqrt[n]{|c_n|} = \frac{|z|}{R}
  \]
  注意:$R$ 被称为 $\Sigma c_n z^n$ 收敛的半径。
\end{proof}

\begin{examples}\mbox{}\par
  \begin{enumerate}[label=(\alph*)]
    \item The series $\Sigma n^n z^n$ has $R = 0$.
    \item The series $\sum\frac{z^n}{n!}$ has $R = +\infty$. (In this case the ratio test is easier to apply
          than the root test.)
    \item The series $\Sigma z^n$ has $R = 1$. If $|z|=1$, the series diverges, since $\{z^n\}$ does not
          tend to $0$ as $n\to\infty$.
    \item The series $\sum\frac{z^n}{n}$ has $R = 1$. It diverges if $z = 1$. It converges for all other $z$
          with $|z| = 1$. (The last assertion will be proved in Theorem 3.44.)
    \item The series $\sum\frac{z^n}{n^2}$ has $R = 1$. It converges for all $z$ with $|z|=1$, by the
          comparison test, since $|z^n/n^2| = 1/n^2$.
  \end{enumerate}
\end{examples}

% TODO

\end{document}
