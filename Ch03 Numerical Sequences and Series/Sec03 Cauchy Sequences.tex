\documentclass[../poma-notes.tex]{subfiles}

\begin{document}

\subsection*{Cauchy Sequences}

\begin{definition}
  A sequnce $\{p_n\}$ in a metric space $X$ is said to be a \textit{Cauchy} sequence if for every $\epsilon > 0$
  there is an integer $N$ such that $d(p_n, p_m) < \epsilon$ if $n \ge N$ and $m \ge N$.
\end{definition}

\begin{definition}
  Let $E$ be a nonempty subset of a metric space $X$, and let $S$ be the set of all real numbers of the form $d(p,q)$,
  with $p \in E$ and $q \in E$. The sup of $S$ is called the \textit{diameter} of $E$.
\end{definition}

如果 $\{p_n\}$ 是 $X$ 中的一个序列,且如果 $E_n$ 包含所有点 $p_N, p_{N+1}, p_{N+2}, \dots$,那么从前两个定义中可以清楚的得到
$\{p_n\}$ 是一个\textbf{柯西序列} 当且仅当
\[\lim_{N \to \infty} diam\ E_N = 0\]

\end{document}
