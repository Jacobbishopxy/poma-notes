\documentclass[../poma-notes.tex]{subfiles}

\begin{document}

\subsection*{Rearrangements}

\begin{definition}
  Let $\{k_n\},\ n=1,2,3,\dots$, be a sequence in which every positive integer appears once and only once
  (that is, $\{k_n\}$ is a 1-1 function from $\pmb{J}$ onto $\pmb{J}$, in the notation of Definition 2.2).
  Putting
  \[
    a'_n = a_{k_n} \qquad (n=1,2,3,\dots)
  \]
  we say that $\Sigma a'_n$ is a \textbf{rearrangement} of $\Sigma a_n$.
\end{definition}

如果 $\{s_n\},\ \{s'_n\}$ 分别为 $\Sigma a_n,\ \Sigma a'_n$ 部分和的序列,那么很容易知道这两个序列有完全不同的成员
所组成。因此问题就被引导到:在什么条件下,一个收敛级数的所有重排会收敛,以及总和是否一定相同。

\begin{example}
  Consider the convergent series
  \begin{equation}
    1 - \frac{1}{2} + \frac{1}{3} - \frac{1}{4} + \frac{1}{5} - \frac{1}{6} + \cdots
  \end{equation}
  and one of its rearrangements
  \begin{equation}
    1 + \frac{1}{3} - \frac{1}{2} + \frac{1}{5} + \frac{1}{7} - \frac{1}{4} + \frac{1}{9} + \frac{1}{11} -
    \frac{1}{6} + \cdots
  \end{equation}
  in which two positive terms are always followed by one negative. If $s$ is the sum of (22), then
  \[
    s < 1 - \frac{1}{2} + \frac{1}{3} = \frac{5}{6}
  \]
  Since
  \[
    \frac{1}{4k - 3} + \frac{1}{4k - 1} - \frac{1}{2k} > 0
  \]
  for $k \ge 1$, we see that $s'_3 < s'_6 < s'_9 < \cdots$, where $s'_n$ is $n$th partial sum of (23).
  Hence
  \[
    \limsup_{n\to\infty} s'_n > s'_3 = \frac{5}{6}
  \]
  so that (23) certainly does not converge to $s$.
\end{example}

\begin{theorem}
  Let $\Sigma a_n$ be a series of real numbers which converges, but not absolutely. Suppose
  \[
    -\infty \le \alpha \le \beta \le \infty
  \]
  Then there exists a rearrangement $\Sigma a'_n$ with partial sums $s'_n$ such that
  \begin{equation}
    \liminf_{n\to\infty} s'_n = \alpha, \qquad \limsup_{n\to\infty} s'_n = \beta.
  \end{equation}
\end{theorem}

\begin{proof}
  令
  \[
    p_n = \frac{|a_n|+a_n}{2},\qquad q_n = \frac{|a_n|-a_n}{2} \qquad (n=1,2,3,\dots)
  \]
  那么 $p_n - q_n = a_n,\ p_n + q_n = |a_n|,\ p_n \ge 0,\ q_n \ge 0$。级数 $\Sigma p_n,\ \Sigma q_n$ 必须皆为
  发散。

  如果皆为收敛,那么
  \[
    \Sigma(p_n + q_n) = \Sigma |a_n|
  \]
  将会收敛,这与假设相悖。又因为
  \[
    \sum_{n=1}^{N} a_n = \sum_{n=1}^{N} (p_n - q_n) = \sum_{n=1}^{N} p_n - \sum_{n=1}^{N} q_n
  \]
  若 $\Sigma p_n$ 发散而 $\Sigma q_n$ 收敛(或者反过来)说明 $\Sigma a_n$ 发散,同样与假设相悖。

  现在令 $P_1,P_2,P_3,\dots$ 顺序代表 $\Sigma a_n$ 的非负项,令 $Q_1,Q_2,Q_3,\dots$ 顺序代表 $\Sigma a_N$
  负数项的绝对值。

  级数 $\Sigma P_n, \Sigma Q_n$ 与 $\Sigma p_n, \Sigma q_n$ 仅不同于为零的项,因此是发散的。

  接着构建序列 $\{m_n\}, \{k_n\}$ 满足级数
  \begin{multline}
    \qquad P_1 + \cdots + P_{m_1} - Q_1 - \cdots - Q_{k_1} + P_{m_1 + 1} + \cdots \\
    + P_{m_2} - Q_{k_1 + 1} - \cdots - Q_{k_2} + \cdots \qquad
  \end{multline}
  即显然是 $\Sigma a_n$ 的重排,满足式 (24)。

  选择实数序列 $\{\alpha_n\},\{\beta_n\}$ 满足 $\alpha_n\to\alpha, \beta_n\to\beta, \alpha_n<\beta, \beta_1>0$。

  令 $m_1, k_1$ 为最小的整数满足
  \begin{gather*}
    P_1 + \cdots + P_{m_1} > \beta_1, \\
    P_1 + \cdots + P_{m_1} - Q_1 - \cdots - Q_{k_1} < \alpha_1;
  \end{gather*}
  令 $m_2, k_2$ 为最小的整数满足
  \begin{gather*}
    P_1 + \cdots + P_{m_1} - Q_1 - \cdots - Q_{k_1} + P_{m_1+1} + \cdots + P_{m_2} > \beta_2, \\
    P_1 + \cdots + P_{m_1} - Q_1 - \cdots - Q_{k_1} + P_{m_1+1} + \cdots + P_{m_2} - Q_{k_1+1} -
    \cdots - Q_{k_2} < \alpha_2;
  \end{gather*}
  并持续这个过程。由于 $\Sigma P_n$ 与 $\Sigma Q_n$ 是发散的,因此上述过程成立。

  如果 $x_n, y_n$ 代表式 (25) 的部分和,其最终项为 $P_{m_n}, -Q_{k_n}$,那么
  \[
    |x_n - \beta_n| \le P_{m_n},\qquad |y_n - \alpha_n| \le Q_{k_n}
  \]
  由于在 $n \to 0$ 时有 $P_n \to 0$ 以及 $Q_n \to 0$,可得 $x_n \to \beta, y_n \to \alpha$。

  最后,很明显没有小于 $\alpha$ 或是大于 $\beta$ 的数可以成为式 (25) 部分和子序列的极限。
\end{proof}

\begin{anote}
  论证关键:
  \begin{enumerate}
    \item 证明 $\Sigma p_n,\ \Sigma q_n$ 皆为发散;
    \item 证明式 (25) 为 $\Sigma a_n$ 的重排;
    \item 不断取 $m,\ k$ 最小整数,重复步骤证明式 (25) 被约束在 $(\alpha, \beta)$ 之内。
  \end{enumerate}

  关于式 (25):
  \begin{align*}
    \begin{split}
      [(P_1 + \cdots) + \aul[red]{P_{m_1}} + (P_{m_1 + 1} + \cdots) + \aul[red]{P_{m_2}} +
        (P_{m_2 + 1} + \cdots) + \cdots] \makebox[2em][l]{\quad{\acircled{1}}} \\
      - [(Q_1 + \cdots) + \aul[blue]{Q_{k_1}} + (Q_{k_1 + 1} + \cdots) + \aul[blue]{Q_{k_2}} +
      (Q_{k_2 + 1} + \cdots) + \cdots] \makebox[2em][l]{\quad{\acircled{2}}}
    \end{split}
  \end{align*}

  显然 $\acircled{1} - \acircled{2}$ 为 $\Sigma a_n$ 的重排。

  那么对于 $m_1,\ k_1$ 而言:
  \begin{align*}
    \aul[red]{(P_1 + \cdots) + P_{m_1}} > \beta_1
  \end{align*}
  \begin{align*}
    [\aul[red]{(P_1 + \cdots) + P_{m_1}}] - [\aul[blue]{(Q_1 + \cdots) + Q_{k_1}}] < \alpha_1
  \end{align*}

  对于 $m_2,\ k_2$ 而言:
  \begin{align*}
    \aul[red]{(P_1 + \cdots) + P_{m_1}} + \aul[red]{(P_{m_1 + 1} + \cdots) + P_{m_2}} > \beta_2
  \end{align*}
  \begin{multline*}
    \qquad [\aul[red]{(P_1 + \cdots) + P_{m_1}} + \aul[red]{(P_{m_1 + 1} + \cdots) + P_{m_2}}] - \\
    [\aul[blue]{(Q_1 + \cdots) + Q_{k_1}} + \aul[blue]{(Q_{k_1+1} + \cdots) + Q_{k_2}}] < \alpha_2 \qquad
  \end{multline*}

  $\Sigma P_n$ 与 $\Sigma Q_n$ 发散,上述过程可一直重复。
\end{anote}

\begin{theorem}
  If $\Sigma a_n$ is a series of complex numbers which converges absolutely, then every rearrangement of
  $\Sigma a_n$ converges, and they all converge to the same sum.
\end{theorem}

\begin{proof}
  令 $\Sigma a'_n$ 作为重排,其部分和为 $s'_n$。给定 $\varepsilon > 0$,存在一个整数 $N$ 使得 $m \ge n \ge N$ 满足
  \begin{equation}
    \sum_{i=n}^{m} |a_i| \le \varepsilon
  \end{equation}
  选择 $p$ 使得整数 $1,2,\dots,N$ 都包含在 $k_1,k_2,\dots,k_p$ 集中(使用 Definition 3.52)。那么如果 $n > p$,
  那么 $a_1,\dots,a_N$ 将取消差值 $s_n - s'_n$,根据 (26) 满足 $|s_n - s'_n| \le \varepsilon$。因此 $\{s'_n\}$
  与 $\{s_n\}$ 一样收敛。
\end{proof}

\end{document}
