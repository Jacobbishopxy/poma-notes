\documentclass[../poma-notes.tex]{subfiles}

\begin{document}

\subsection*{The Derivative of a Real Function}

本章我们将着重于定义在区间或者线段上的\textit{实}函数(除了最后一节)。这不仅只是便利的缘故,因为当从实函数传递到向量值函数时,
真正的差异就会出现。定义在 $R^k$ 上的函数微分将在第九章进行讨论。

\begin{definition}
  Let $f$ be defined (and real-valued) on $[a, b]$. For any $x \in [a,b]$ form the quotient
  \begin{equation}
    \phi(t) = \frac{f(t) - f(x)}{t - x} \qquad (a<t<b,\ t \ne x),
  \end{equation}
  and define
  \begin{equation}
    f'(x) = \lim_{t \to x} \phi(t),
  \end{equation}
  provided this limit exists in accordance with Definition 4.1.

  We thus associate with the function $f$ a function $f'$ whose domain is the set of points $x$ at which the
  limit (2) exists; $f'$ is called the \textit{derivative} of $f$.

  If $f'$ is defined at a point $x$, we say that $f$ is \textit{differentiable} at $x$. If $f'$ is defined at
  every point of a set $E \subset [a,b]$, we say that $f$ is differentiable on $E$.

  It is possible to consider right-hand and left-hand limits in (2); this leads to the definition of right-hand
  and left-hand derivatives. In particular, at the endpoints $a$ and $b$, the derivative, if it exists, is a
  right-hand of left-hand derivative, respectively. We shall not, however, discuss one-sided derivatives in
  any detail.

  If $f$ is defined on a segment $(a, b)$ and if $a<x<b$, then $f'(x)$ is defined by (1) and (2), as above.
  But $f'(a)$ and $f'(b)$ are not defined in this case.
\end{definition}

\begin{anote}
  \textbf{导数(导函数)}:定义在 $[a, b]$ 上的实值函数,任何 $x \in [a, b]$ 其商形式 $\phi(t)=\frac{f(t)-f(x)}{t-x}
    \ (a < t < b,\ t \ne x)$,且 $f'$ 在点 $x$ 满足定义 $f'(x) = \lim_{t \to x} \phi(t)$,那么 $f$ 在 $x$ 上
  \textbf{可微} 或 \textbf{可导}。
\end{anote}

\begin{theorem}
  Let $f$ be defined on $[a, b]$. If $f$ is differentiable at a point $x \in [a, b]$, then $f$ is continuous
  at $x$.
\end{theorem}

\begin{proof}
  根据 Theorem 4.4,在 $t \to x$ 时,有
  \[
    f(t) - f(x) = \frac{f(t) - f(x)}{t - x} \cdot (t - x) \to f'(x) \cdot 0 = 0
  \]
\end{proof}

\begin{anote}
  若 $f$ 在 $x \in [a, b]$ 可微(可导),那么它在 $x$ 点连续。
\end{anote}

本定理反过来并不成立。构建一个连续函数在其独立点上不可微是很容易的。在第七章节中,我们应该会更加熟悉一个函数在整条线是连续的,但是
在任意一点上皆不可微!

\begin{theorem}
  Suppose $f$ and $g$ are defined on $[a, b]$ and are differentiable at a point $x \in [a,b]$. Then $f + g$,
  $fg$, and $f/g$ are differentiable at $x$, and
  \begin{enumerate}[label=(\alph*)]
    \item $(f+g)'(x) = f'(x) + g'(x)$;
    \item $(fg)'(x) = f'(x)g(x) + f(x)g'(x)$;
    \item $(\frac{f}{g})'(x) = \frac{g(x)f'(x) - g'(x)f(x)}{g^2(x)}$.
  \end{enumerate}
  In (c), we assume of course that $g(x) \ne 0$.
\end{theorem}

\begin{proof}\mbox{}\par
  根据 Theorem 4.4 可证 (a)。令 $h = fg$,那么
  \[
    h(t) - h(x) = f(t)[g(t) - g(x)] + g(x)[f(t) - f(x)]
  \]
  如果将上式除以 $t - x$,且根据 Theorem 5.2,$t \to x$ 时有 $f(t) \to f(x)$ ,可得 (b)。接下来就是另 $h = f/g$。那么
  \[
    \frac{h(t) - h(x)}{t - x} = \frac{1}{g(t)g(x)} [g(x)\frac{f(t)-f(x)}{t-x} - f(x)\frac{g(t)-g(x)}{t-x}]
  \]
  令 $t \to x$,并应用 Theorems 4.4 与 5.2,可得 (c)。
\end{proof}

\stepcounter{poma}

\begin{theorem}
  Suppose $f$ is continuous on $[a, b]$, $f'(x)$ exists at some point $x \in [a, b]$, $g$ is defined on an
  interval $I$ which contains the range of $f$, and $g$ is differentiable at the point $f(x)$. If
  \[
    h(t) = g(f(t)) \qquad (a \le t \le b),
  \]
  then $h$ is differentiable at $x$, and
  \begin{equation}
    h'(x) = g'(f(x)) f'(x).
  \end{equation}
\end{theorem}

\begin{proof}
  令 $y = f(x)$,根据微分的定义有
  \begin{align}
    f(t) - f(x) = (t - x)[f'(x) + u(t)] \\
    g(s) - g(y) = (s - y)[g'(y) + y(s)]
  \end{align}
  其中 $t \in [a,b],\ s \in I$,且 $t \to x$ 有 $u(t) \to 0$,$s \to y$ 有 $v(s) \to 0$。令 $s = f(t)$,通过 (5)
  再 (4),可得
  \begin{align*}
    \begin{split}
      h(t) - h(x) & = g(f(t)) - g(f(x)) \\
      & = [f(t) - f(x)] \cdot [g'(y) + v(s)] \\
      & = (t - x) \cdot [f'(x) + u(t)] \cdot [g'(y) + v(s)]
    \end{split}
  \end{align*}
  或者如果 $t \ne x$,
  \begin{equation}
    \frac{h(t) - h(x)}{t - x} = [g'(y) + v(s)] \cdot [f'(x) + u(t)]
  \end{equation}
  当 $t \to x$,可知 $s \to y$,根据 $f$ 的连续性,(6) 等式右侧趋向于 $g'(y) f'(x)$,即得 (3)。
\end{proof}

\begin{examples}
  \begin{enumerate}[label=(\alph*)]
    \item 令 $f$ 的定义如下
          \begin{equation}
            f(x) =
            \begin{cases}
              x \sin \frac{1}{x} & (x \ne 0), \\
              0                  & (x = 0).
            \end{cases}
          \end{equation}

          理所当然的 $\sin x$ 的导数是 $\cos x$ (我们将在第八章对三角函数进行讨论),将 Theorems 5.3 与 5.5 应用在 $f$ 非零部分,有
          \begin{equation}
            f'(x) = \sin\frac{1}{x} - \frac{1}{x}\cos\frac{1}{x} \qquad (x \ne 0)
          \end{equation}
          在 $x = 0$ 时,这些定理不再适用,因为 $1/x$ 在零时没有定义。直接给出定义:对于 $t \ne 0$,
          \[
            \frac{f(t) - f(0)}{t - 0} = \sin\frac{1}{5}
          \]
          当 $x \to 0$ 时,它没有趋向任何极限,因此 $f'(0)$ 不存在。
    \item 令 $f$ 的定义如下
          \begin{equation}
            f(x) =
            \begin{cases}
              x^2 \sin\frac{1}{x} & (x \ne 0), \\
              0                   & (x = 0)
            \end{cases}
          \end{equation}

          与 (a) 类似,有
          \begin{equation}
            f'(x) = 2x \sin\frac{1}{x} - \cos\frac{1}{x} \qquad (x \ne 0)
          \end{equation}
          在 $x = 0$ 时,给出定义
          \[
            \biggl| \frac{f(t) - f(0)}{t - 0} \biggr| = \biggl| t \sin\frac{1}{t} \biggr| \le |t| \qquad (t \ne 0)
          \]
          令 $t \to 0$,可得
          \begin{equation}
            f'(0) = 0
          \end{equation}
  \end{enumerate}

  因此 $f$ 在所有点 $x$ 上可导,但是 $f'$ 并不是一个连续函数,因为 $\cos(1/x)$ 在 (10) 中 $x \to 0$ 时,并没有趋向与一个极限。
\end{examples}

\end{document}

