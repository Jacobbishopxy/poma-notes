\documentclass[../poma-notes.tex]{subfiles}

\begin{document}

\subsection*{Differentiation of Vector-valued Functions}

\begin{remarks}
  Definition 5.1 applies without any change to complex functions $f$ defined on $[a,b]$, and Theorems 5.2 and 5.3,
  as well as their proofs, remain valid. If $f_1$ and $f_2$ are the real and imaginary parts of $f$, that is, if
  \[
    f(t) = f_1(t) + i f_2(t)
  \]
  for $a \le t \le b$, where $f_1(t)$ and $f_2(t)$ are real, then we clearly have
  \begin{equation}
    f'(x) = f_1'(x) + i f_2'(x);
  \end{equation}
  also, $f$ is differentiable at $x$ if and only if both $f_1$ and $f_2$ are differentiable at $x$.

  Passing to vector-valued functions in general, i.e., to functions $\mathbf{f}$ which map $[a,b]$ into some $R^k$,
  we may still apply Definition 5.1 to define $\mathbf{f}'(x)$. The term $\phi(t)$ in (1) is now, for each $t$,
  a point in $R^k$, and the limit in (2) is taken with respect to the norm of $R^k$. In other words, $\mathbf{f}'(x)$
  is that point of $R^k$ (if there is one) for which
  \begin{equation}
    \lim_{t\to x} \biggl| \frac{\mathbf{f}(t) - \mathbf{f}(x)}{t - x} - \mathbf{f}'(x) \biggr| = 0
  \end{equation}
  and $\mathbf{f}'$ is again a function with values in $R^k$.

  If $f_1,\dots,f_k$ are the components of $\mathbf{f}$, as defined in Theorem 4.10, then
  \begin{equation}
    \mathbf{f}' = (f_1',\cdots,f_k'),
  \end{equation}
  and $\mathbf{f}$ is differentiable at a point $x$ if and only if each of the functions $f_1,\cdots,f_k$ is
  differentiable at $x$.

  Theorem 5.2 is true in this context as well, and so is Theorem 5.3(a) and (b), if $fg$ is replaced by the inner
  product $\mathbf{f \cdot g}$ (see Definition 4.3).

  When we turn to the mean value theorem, however, and to one of its consequences, namely, L'Hospital's rule,
  the situation changes. The next two examples will show that each of these results fails to be true for
  complex-valued functions.
\end{remarks}

% TODO

\end{document}
