\documentclass[../poma-notes.tex]{subfiles}

\begin{document}

\subsection*{Differentiation of Vector-valued Functions}

\begin{remarks}
  Definition 5.1 applies without any change to complex functions $f$ defined on $[a,b]$, and Theorems 5.2 and 5.3,
  as well as their proofs, remain valid. If $f_1$ and $f_2$ are the real and imaginary parts of $f$, that is, if
  \[
    f(t) = f_1(t) + i f_2(t)
  \]
  for $a \le t \le b$, where $f_1(t)$ and $f_2(t)$ are real, then we clearly have
  \begin{equation}
    f'(x) = f_1'(x) + i f_2'(x);
  \end{equation}
  also, $f$ is differentiable at $x$ if and only if both $f_1$ and $f_2$ are differentiable at $x$.

  Passing to vector-valued functions in general, i.e., to functions $\mathbf{f}$ which map $[a,b]$ into some $R^k$,
  we may still apply Definition 5.1 to define $\mathbf{f}'(x)$. The term $\phi(t)$ in (1) is now, for each $t$,
  a point in $R^k$, and the limit in (2) is taken with respect to the norm of $R^k$. In other words, $\mathbf{f}'(x)$
  is that point of $R^k$ (if there is one) for which
  \begin{equation}
    \lim_{t\to x} \biggl| \frac{\mathbf{f}(t) - \mathbf{f}(x)}{t - x} - \mathbf{f}'(x) \biggr| = 0
  \end{equation}
  and $\mathbf{f}'$ is again a function with values in $R^k$.

  If $f_1,\dots,f_k$ are the components of $\mathbf{f}$, as defined in Theorem 4.10, then
  \begin{equation}
    \mathbf{f}' = (f_1',\cdots,f_k'),
  \end{equation}
  and $\mathbf{f}$ is differentiable at a point $x$ if and only if each of the functions $f_1,\cdots,f_k$ is
  differentiable at $x$.

  Theorem 5.2 is true in this context as well, and so is Theorem 5.3(a) and (b), if $fg$ is replaced by the inner
  product $\mathbf{f \cdot g}$ (see Definition 4.3).

  When we turn to the mean value theorem, however, and to one of its consequences, namely, L'Hospital's rule,
  the situation changes. The next two examples will show that each of these results fails to be true for
  complex-valued functions.
\end{remarks}

\begin{example}
  Define, for real $x$,
  \begin{equation}
    f(x) = e^{ix} = \cos x + i \sin x.
  \end{equation}
  (The last expression may be taken as the definition of the complex exponential $e^{ix}$; see Chap. 8 for a full
  discussion of these functions.) Then
  \begin{equation}
    f(2\pi) - f(0) = 1 - 1 = 0
  \end{equation}
  but
  \begin{equation}
    f'(x) = i e^{ix},
  \end{equation}
  so that $|f'(x)| = 1$ for all real $x$.

  Thus Theorem 5.10 fails to hold in this case.
\end{example}

\begin{example}
  On the segment $(0,1)$, define $f(x) = x$ and
  \begin{equation}
    g(x) = x + x^2 e^{i/x^2}.
  \end{equation}
  Since $|e^{it}| = 1$ for all real $t$, we see that
  \begin{equation}
    \lim_{x\to 0} \frac{f(x)}{g(x)} = 1.
  \end{equation}
  Next,
  \begin{equation}
    g'(x) = 1 + \biggl\{ 2x - \frac{2i}{x} \biggr\}^{i/x^2} \qquad (0<x<1),
  \end{equation}
  so that
  \begin{equation}
    |g'(x)| \ge \biggl\| 2x - \frac{2i}{x} \biggr\| - 1 \ge \frac{2}{x} - 1.
  \end{equation}
  Hence
  \begin{equation}
    \biggl| \frac{f'(x)}{g'(x)} \biggr| = \frac{1}{|g'(x)|} \le \frac{x}{2-x}
  \end{equation}
  and so
  \begin{equation}
    \lim_{x\to 0} \frac{f'(x)}{g'(x)} = 0.
  \end{equation}
  By (36) and (40), L'Hospital's rule fails in this case. Note also that $g'(x) \ne 0$ on $(0,1)$, by (38).

  However, there is a consequence of the mean value theorem which, for purposes of applications, is almost as useful
  as Theorem 5.10, and which remains true for vector-valued functions: From Theorem 5.10 it follows that
  \begin{equation}
    |f(b) - f(a)| \le (b-a) \sup_{a<x<b} |f'(x)|.
  \end{equation}
\end{example}

\begin{theorem}
  Suppose $\mathbf{f}$ is a continuous mapping of $[a, b]$ into $R^k$ and $\mathbf{f}$ is differentiable in $(a, b)$. Then
  there exists $x \in (a, b)$ such that
  \[
    |\mathbf{f}(b) - \mathbf{f}(a)| \le (b - a) |\mathbf{f}'(x)|.
  \]
\end{theorem}

\begin{proof}
  令 $\mathbf{z} = \mathbf{f}(b) - \mathbf{f}(a) - \mathbf{f}'(b)$,同时定义
  \[
    \phi(t) = \mathbf{z} \cdot \mathbf{f}(t) \qquad (a \le t \le b)
  \]
  那么 $\phi$ 在区间 $[a, b]$ 上为一个实值连续函数,且在 $(a, b)$ 上可微。根据中值定理有
  \[
    \phi(b) - \phi(a) = (b - a) \phi'(x) = (b - a) \mathbf{z} \cdot \mathbf{f}'(x)
  \]
  其中 $x \in (a, b)$。另一方面,
  \[
    \phi(b) - \phi(a) = \mathbf{z} \cdot \mathbf{f}(b) - \mathbf{z} \cdot \mathbf{f}(a) = \mathbf{z} \cdot \mathbf{z} = |\mathbf{z}|^2
  \]

  根据 Schwarz 不等式可得
  \[
    |\mathbf{z}|^2 = (b-a) |\mathbf{z} \cdot \mathbf{f}'(x)| \le (b - a)|\mathbf{z}||\mathbf{f}'(x)|
  \]
  因此 $|\mathbf{z}| \le (b-a)|\mathbf{f}'(x)|$,即期望的结论。
\end{proof}

\begin{anote}
  设 $\mathbf{f}$ 是把 $[a, b]$ 映入 $R^k$ 内的连续映射,并且 $\mathbf{f}$ 在 $(a, b)$ 内可微。那么必有 $x \in (a,b)$ 使得
  $|\mathbf{f}(b) - \mathbf{f}(a)| \le (b-a)|\mathbf{f}'(x)|$。
\end{anote}

\end{document}
