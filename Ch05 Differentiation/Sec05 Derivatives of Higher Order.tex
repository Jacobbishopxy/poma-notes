\documentclass[../poma-notes.tex]{subfiles}

\begin{document}

\subsection*{Derivatives of Higher Order}

\begin{definition}
  If $f$ has a derivative $f'$ on an interval, and if $f'$ is itself differentiable, we denote the derivative of
  $f'$ by $f''$ and cal $f''$ the second derivative of $f$. Continuing in this manner, we obtain functions
  \[
    f,f',f'',f^(3),\cdots,f^(n),
  \]
  each of which is the derivative of the preceding one. $f^(n)$ is called the $n$th derivative, or the derivative
  of order $n$, of $f$.

  In order for $f^(n)(x)$ to exist at a point $x$, $f^(n-1)(t)$ must exist in a neighborhood of $x$ (or in a
  one-sided neighborhood, if $x$ is an endpoint of the interval on which $f$ is defined), and $f^(n-1)$ must be
  differentiable at $x$. Since $f^(n-1)$ must exist in a neighborhood of $x$, $f^(n-2)$ must be differentiable
  in that neighborhood.
\end{definition}

\begin{anote}
  如果 $f$ 在一个区间由导数 $f'$,而 $f'$ 自身又是可微的,把 $f'$ 的导数记为 $f''$,叫做\textbf{二阶导数}。这样继续下去得到
  $f, f', f'', f^(3), \cdots, f^(n)$,其中 $f^(n)$ 叫做 $n$ 阶导数。
\end{anote}

\end{document}
