\documentclass[../poma-notes.tex]{subfiles}

\begin{document}

\subsection*{Monotonic Functions}

现在开始学习在给定线段上永不减少(或增加)的函数。

\begin{definition}
  Let $f$ be real on $(a, b)$. Then $f$ is said to be \textit{monotonically increasing} on $(a, b)$ if
  $a < x < y < b$ implies $f(x) \le f(y)$. If the last inequality is reversed, we obtain the definition of
  a \textit{monotonically decreasing} function. The class of monotonic functions consists of both the
  increasing and the decreasing functions.
\end{definition}

\begin{anote}
  设 $f:\ (a, b) \to R$,若 $a < x < y < b$ 时有 $f(x) \le f(y)$,那么 $f$ 在 $(a, b)$ 上 \textbf{单调递增};
  若有 $f(x) \le f(y)$ 就是 \textbf{单调递减}。两者统称为 \textbf{单调函数}。
\end{anote}

\begin{theorem}
  Let $f$ be monotonically increasing on (a, b). Then $f(x+)$ and $f(x-)$ exist at every point of $x$ of $(a, b)$.
  More precisely,
  \begin{equation}
    \sup_{a<t<x} f(t) = f(x-) \le f(x) \le f(x+) = \inf_{x<t<b} f(t).
  \end{equation}
  Furthermore, if $a < x < y < b$, then
  \begin{equation}
    f(x+) \le f(y-).
  \end{equation}
  Analogous results evidently hold for monotonically decreasing functions.
\end{theorem}

\begin{proof}
  根据假设,$f(t)$ 的集合有 $a < t < x$,其上界为 $f(x)$,因此有一个最小上界 $A$。显然 $A \le f(x)$,现在需要证明 $A = f(x-)$。

  给定 $\varepsilon > 0$。遵循 $A$ 的最小上界定义,存在 $\delta > 0$ 满足 $a < x - \delta < x$ 且
  \begin{equation}
    A - \varepsilon < f(x - \delta) \le A
  \end{equation}

  由于 $f$ 是单调的,有
  \begin{equation}
    f(x - \delta) \le f(t) \le A \qquad (x - \delta < t < x)
  \end{equation}

  结合 (27) 与 (28),可得
  \[
    |f(t) - A| < \varepsilon \qquad (x - \delta < t < x)
  \]
  因此 $f(x-) = A$。

  同理可得 (25) 的另一部分。

  接着,如果 $a < x < y < b$,从 (25) 可得
  \begin{equation}
    f(x+) = \inf_{x<t<b} f(t) = \inf_{x<t<y} f(t)
  \end{equation}

  最后将 (25) 应用在 $(a, y)$ 替换 $(a, b)$。同样的,
  \begin{equation}
    f(y-) = \sup_{a<t<y} f(t) = \sum_{x<t<y} f(t)
  \end{equation}

  比较 (29) 与 (30),得 (26)。
\end{proof}

\begin{corollary}
  Monotonic functions have no discontinuities of the second kind.
\end{corollary}

\begin{anote}
  单调函数没有第二类间断。
\end{anote}

\begin{theorem}
  Let $f$ be monotonic on $(a, b)$. Then the set of points of $(a, b)$ at which $f$ is discontinuous is at most countable.
\end{theorem}

\begin{proof}
  假设为了定义的目的,令 $f$ 递增,同时令 $E$ 为 $f$ 所有不连续点的集合。

  将 $E$ 的每个 $x$ 点关联一个实数 $r(x)$ 满足
  \[
    f(x-) < r(x) < f(x+)
  \]
  由于 $x_1 < x_2$ 得 $f(x_1+) \le f(x_2-)$,可得 $r(x_1) \ne r(x_2)$ 如果 $x_1 \ne x_2$。

  由此为 $E$ 与实数的子集建立了一对一关系,即得 $E$ 可数。
\end{proof}

\begin{anote}
  设 $f$ 在 $(a, b)$ 上单调,那么 $(a, b)$ 中使 $f$ 间断的点最多是可数的。
\end{anote}

\begin{remark}
  It should be noted that the discontinuities of a monotonic function need not be isolated. In fact, given any countable
  subset $E$ of $(a, b)$, which may even be dense, we can construct a function $f$, monotonic on $(a, b)$, discontinuous at
  every point of $E$, and at no other point of $(a, b)$.

  To show this, let the points of $E$ be arranged in a sequence $\{x_n\},\ n=1,2,3,\dots$. Let $\{c_n\}$ be a sequence of
  positive numbers such that $\Sigma c_n$ converges. Define
  \begin{equation}
    f(x) = \sum_{x_n < x} c_n \qquad (a<x<b)
  \end{equation}

  The summation is to be understood as follows: Sum over those indices $n$ for which $x_n < x$. If there are no points $x_n$
  to the left of $x$, the sum is empty; following the usual convention, we define it to be zero. Since (31) converges
  absolutely, the order in which the terms are arranged is immaterial.

  We leave the verification of the following properties of $f$ to the reader:
  \begin{enumerate}[label=(\alph*)]
    \item $f$ is monotonically increasing on $(a, b)$;
    \item $f$ is discontinuous at every point of $E$; in fact,
          \[
            f(x_n+) - f(x_n-) = c_n.
          \]
    \item $f$ is continuous at every other point of $(a, b)$.
  \end{enumerate}

  Moreover, it is not hard to see that $f(x-) = f(x)$ at all points of $(a, b)$. If a function satisfies this condition, we say
  that $f$ is \textit{continuous from the left}. If the summation in $(31)$ were taken over all indices $n$ for which $x_n \le x$,
  we would have $f(x+) = f(x)$ at every point of $(a, b)$; that is, $f$ would be \textit{continuous from the right}.

  Functions of this sort can also be defined by another method; for an example we refer to Theorem 6.16.
\end{remark}

\begin{anote}
  间断点不一定是孤立点。
\end{anote}

\end{document}

