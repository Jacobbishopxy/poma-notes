\documentclass[../poma-note.tex]{subfiles}

\begin{document}

\subsection*{Finite, Countable, and Uncoutable Sets}

本节由函数概念的定义开始。

\begin{definition}
  Consider two sets $A$ and $B$, whose elements may be any objects whatsover, and suppose that with each
  element $x$ of $A$ there is associated, in some manner, an element of $B$, which we denote by $f(x)$.
  Then $f$ is said to be a \textit{function} from $A$ to $B$ (or a \textit{mapping} of $A$ into $B$).
  The set $A$ is called the \textit{domain} of $f$ (we also say $f$ is defined on $A$), and the elements $f(x)$
  are called the \textit{values} of $f$. The set of all values of $f$ is called the \textit{range} of $f$.
\end{definition}

\begin{definition}
  Let $A$ and $B$ be two sets and let $f$ be a mapping of $A$ into $B$. If $E \subset A$, $f(E)$ is defined to
  be the set of all elements $f(x)$, for $x \in E$. We call $f(E)$ the \textit{image} of $E$ under $f$.
  In this notation, $f(A)$ is the range of $f$. It is clear that $f(A) \subset B$. If $f(A)=B$, we say that $f$
  maps $A$ \textit{onto} $B$. (Note that, according to this usage, \textit{onto} is more specific than \textit{into}.)

  If $E \subset B$, $f^{-1}(E)$ denotes the set of all $x \in A$ such that $f(x) \in E$. We call $f^{-1}(E)$ the
  \textit{inverse image} of $E$ under $f$. If $y \in B$, $f^{-1}(y)$ is the set of all $x \in A$ such that
  $f(x)=y$. If, for each $y \in B$, $f^{-1}(y)$ consists of at most one element of $A$, then $f$ is said to be
  a 1-1 (\textit{one-to-one}) mapping of $A$ into $B$. This may also be expressed as follows: $f$ is a 1-1 mapping
  of $A$ into $B$ provided that $f(x_1) \ne f(x_2)$ whenever $x_1 \ne x_2,\ x_1 \in A,\ x_2 \in A$.

  (The notation $x_1 \ne x_2$ means that $x_1$ and $x_2$ are distinct elements; otherwise we write $x_1=x_2$.)
\end{definition}

\begin{definition}
  If there exists a 1-1 mapping of $A$ \textit{onto} $B$, we say that $A$ and $B$ can be put in 1-1 \textit{correspondence},
  or that $A$ and $B$ have the same \textit{cardinal number}, or, briefly, that $A$ and $B$ are \textit{equivalent},
  and we write $A \sim B$. This relation clearly has the following properties:
  \begin{itemize}
    \item[] It is relexive: $A \sim A$.
    \item[] It is symmetric: If $A \sim B$, then $B \sim A$.
    \item[] It is transitive: If $A \sim B$ and $B \sim C$, then $A \sim C$.
  \end{itemize}
  Any relation with these three properties is called an \textit{equivalence relation}.
\end{definition}

\begin{definition}
  For any positive integer $n$, let $J_n$ be the set whose elements are the integers $1,2,\dots,n$; let $J$ be
  the set consisting of all positive integers. For any set $A$, we say:
  \begin{enumerate}[label=(\alph*)]
    \item $A$ is \textit{finite} if $A \sim J_n$ for some $n$ (the empty set is also considered to be finnite).
    \item $A$ is \textit{} if $A$ is not finite.
    \item $A$ is \textit{} if $A \sim J$.
    \item $A$ is \textit{} if $A$ is neither finite nor countable.
    \item $A$ is \textit{} if $A$ is finite or countable.
  \end{enumerate}

  Countable sets are sometimes called \textit{enumerable}, or \textit{denumerable}.

  For two finite sets $A$ and $B$, we evidently have $A \sim B$ if and only if $A$ and $B$ contain the same number of
  elements. For infinite sets, however, the idea of \say{having the same number of elements} becomes quite vague,
  whereas the notion of 1-1 correspondence retains its clarity.
\end{definition}

\subsection*{Metric Spaces}

WIP

\subsection*{Compact Sets}

WIP

\subsection*{Perfect Sets}

WIP

\subsection*{Connected Sets}

WIP

\end{document}
