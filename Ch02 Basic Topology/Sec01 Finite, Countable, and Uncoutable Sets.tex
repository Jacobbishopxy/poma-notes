\documentclass[../poma-notes.tex]{subfiles}

\graphicspath{\subfix{../images/}}

\begin{document}

\subsection*{Finite, Countable, and Uncoutable Sets}

本节由函数概念的定义开始。

\begin{definition}
  Consider two sets $A$ and $B$, whose elements may be any objects whatsover, and suppose that with each
  element $x$ of $A$ there is associated, in some manner, an element of $B$, which we denote by $f(x)$.
  Then $f$ is said to be a \textit{function} from $A$ to $B$ (or a \textit{mapping} of $A$ into $B$).
  The set $A$ is called the \textit{domain} of $f$ (we also say $f$ is defined on $A$), and the elements $f(x)$
  are called the \textit{values} of $f$. The set of all values of $f$ is called the \textit{range} of $f$.
\end{definition}

\begin{definition}
  Let $A$ and $B$ be two sets and let $f$ be a mapping of $A$ into $B$. If $E \subset A$, $f(E)$ is defined to
  be the set of all elements $f(x)$, for $x \in E$. We call $f(E)$ the \textit{image} of $E$ under $f$.
  In this notation, $f(A)$ is the range of $f$. It is clear that $f(A) \subset B$. If $f(A)=B$, we say that $f$
  maps $A$ \textit{onto} $B$. (Note that, according to this usage, \textit{onto} is more specific than \textit{into}.)

  If $E \subset B$, $f^{-1}(E)$ denotes the set of all $x \in A$ such that $f(x) \in E$. We call $f^{-1}(E)$ the
  \textit{inverse image} of $E$ under $f$. If $y \in B$, $f^{-1}(y)$ is the set of all $x \in A$ such that
  $f(x)=y$. If, for each $y \in B$, $f^{-1}(y)$ consists of at most one element of $A$, then $f$ is said to be
  a 1-1 (\textit{one-to-one}) mapping of $A$ into $B$. This may also be expressed as follows: $f$ is a 1-1 mapping
  of $A$ into $B$ provided that $f(x_1) \ne f(x_2)$ whenever $x_1 \ne x_2,\ x_1 \in A,\ x_2 \in A$.

  (The notation $x_1 \ne x_2$ means that $x_1$ and $x_2$ are distinct elements; otherwise we write $x_1=x_2$.)
\end{definition}

\begin{anote}
  简单来说:
  \begin{enumerate}
    \item $f(E)$ 是集合 $E$ 通过 $f$ 得到的像(image)。
    \item $f(A)$ 是 $f$ 的范围。
    \item 如果 $f(A)=B$,那么称 $f$ 将 $A$ 完全映射至(onto)$B$ 。
    \item 而当 $E \subset B$ 且 $x \in A$ 时,反函数 $f^{-1}(E)$ 是集合 $E$ 通过 $f$ 得到的反像(inverse image)。
    \item $\forall x \in A$ 通过 $f$ 映射后且满足 $\forall f(x) \in B$ 被称为一一映射(1-1 mapping)。
  \end{enumerate}
\end{anote}

\begin{definition}
  If there exists a 1-1 mapping of $A$ \textit{onto} $B$, we say that $A$ and $B$ can be put in 1-1 \textit{correspondence},
  or that $A$ and $B$ have the same \textit{cardinal number}, or, briefly, that $A$ and $B$ are \textit{equivalent},
  and we write $A \sim B$. This relation clearly has the following properties:
  \begin{itemize}
    \item[] It is reflexive: $A \sim A$.
    \item[] It is symmetric: If $A \sim B$, then $B \sim A$.
    \item[] It is transitive: If $A \sim B$ and $B \sim C$, then $A \sim C$.
  \end{itemize}
  Any relation with these three properties is called an \textit{equivalence relation}.
\end{definition}

\begin{anote}
  \begin{itemize}
    \item 基数 Cardinal number。
    \item 自反性 Reflexive,\href{https://en.wikipedia.org/wiki/Reflexive_relation}{维基百科};
    \item 对称性 Symmetric;
    \item 传递性 Transitive。
  \end{itemize}
\end{anote}

\begin{definition}
  For any positive integer $n$, let $J_n$ be the set whose elements are the integers $1,2,\dots,n$; let $J$ be
  the set consisting of all positive integers. For any set $A$, we say:
  \begin{enumerate}[label=(\alph*)]
    \item $A$ is \textit{finite} if $A \sim J_n$ for some $n$ (the empty set is also considered to be finnite).
    \item $A$ is \textit{infinite} if $A$ is not finite.
    \item $A$ is \textit{countable} if $A \sim J$.
    \item $A$ is \textit{uncountable} if $A$ is neither finite nor countable.
    \item $A$ is \textit{at most countable} if $A$ is finite or countable.
  \end{enumerate}

  Countable sets are sometimes called \textit{enumerable}, or \textit{denumerable}.

  For two finite sets $A$ and $B$, we evidently have $A \sim B$ if and only if $A$ and $B$ contain the same number of
  elements. For infinite sets, however, the idea of \say{having the same number of elements} becomes quite vague,
  whereas the notion of 1-1 correspondence retains its clarity.
\end{definition}

\anote
可数集(countable set),是指每个元素都能与自然数集 $N$ 的每个元素之间能建立一一对应的集合;不可数集顾名思义就是无法与自然数集
$N$ 建立一一对应的集合;至多可数集(at most coutable)是有限集(finite)与可数集(coutable)的统称。

\setcounter{poma}{6}

\begin{definition}
  By a \textit{sequence}, we mean a function $f$ defined on the set $J$ of all positive integers. If $f(n)=x_n$,
  for $n \in J$, it is customary to denote the sequence $f$ by the symbol $\{x_n\}$, or sometimes by $x_1,x_2,x_3,\dots$.
  The values of $f$, that is, the elements $x_n$, are called the \textit{terms} of the sequence. If $A$ is a set and
  if $x_n \in A$ for all $n \in J$, then $\{x_n\}$ is said to be a \textit{sequence} in $A$, or a \textit{sequnce of elements}
  of $A$.
\end{definition}

注意一个数列的 $x_1,x_2,x_3,\dots$ 项不需要是独特的。

由于每个可数集合是一个定义在 $J$ 上一一映射的范围,可将每个可数集合视为一系列不同项的范围。更宽泛来说,任何可数集合中的原始可以被
“排列在一个数列上”。

有时可以将定义中的 $J$ 替换为所有非负整数集合,这样可能会更加的方便,例如开始于 0 而不是 1。

\begin{theorem}
  Every infinite subset of a countable set $A$ is coutable.
\end{theorem}

\begin{proof}
  假设 $E \subset A$,且 $E$ 为无限的。排列 $A$ 中的元素 $x$ 构建 $\{x_n\}$ 独特数列。构建一个满足如下的数列 $\{n_k\}$:

  令 $n_1$ 为最小的正整数使得 $x_{n_1} \in E$。选择 $n_1,\dots,n_{k-1} \ (k=2,3,4,\dots)$,令 $n_k$ 为最小的大于 $n_{k-1}$
  的整数使得 $x_{n_k} \in E$。

  令 $f(k)=x_{n_k} \ (k=1,2,3,\dots)$,我们获取一个 $E$ 与 $J$ 的一一映射关系。

  根据定理,粗略的说可数集合表示了\say{最小的}无限性:没有不可数集合可以成为一个可数集合的子集。
\end{proof}

\anote
一个可数集合 $A$ 的任意无限子集都是可数的。

\begin{definition}
  Let $A$ and $\Omega$ be sets, and suppose that with each element $\alpha$ of $A$ there is associated a subset of
  $\Omega$ which we denote by $E_{\alpha}$.

  The set whose elements are the sets $E_{\alpha}$ will be denoted by $\{E_{\alpha}\}$. Instead of speaking of sets
  of sets, we shall sometimes speak of a collection of sets, or a family of sets.

  The \textit{union} of the sets $E_{\alpha}$ is defined to be the set $S$ such that $x \in S$ if and only if
  $x \in E_{\alpha}$ for at least one $\alpha \in A$. We use the notation
  \begin{equation}
    S = \bigcup \limits_{\alpha\in A} E_{\alpha}.
  \end{equation}
  If $A$ consists of the integers $1,2,\dots,n$, one usually writes
  \begin{equation}
    S = \bigcup \limits_{m=1}^n E_m
  \end{equation}
  or
  \begin{equation}
    S = E_1 \cup E_2 \cup \cdots \cup E_n.
  \end{equation}
  If $A$ is the set of all positive integers, the usual notation is
  \begin{equation}
    S = \bigcup \limits_{m=1}^{\infty} E_m.
  \end{equation}

  The symbol $\infty$ in (4) merely indicates that the union of a \textit{countable} collection of sets is taken,
  and should not be confused with the symbols $+\infty$, $-\infty$, introduced in Definition 1.23.

  The \textit{intersection} of the sets $E_{\alpha}$ is defined to be the set $P$ such that $x \in P$ if and only if
  $x \in E_{\alpha}$ for every $\alpha \in A$. We use the notation
  \begin{equation}
    P = \bigcap \limits_{\alpha \in A} E_{\alpha},
  \end{equation}
  or
  \begin{equation}
    P = \bigcap \limits_{m=1}^n E_m = E_1 \cap E_2 \cap \cdots \cap E_n,
  \end{equation}
  or
  \begin{equation}
    P = \bigcap \limits_{m=1}^{\infty} E_m,
  \end{equation}
  as for unions. If $A \cap B$ is not empty, we say that $A$ and $B$ \textit{intersect}; otherwise they are \textit{disjoint}.
\end{definition}

\begin{anote}
  \begin{itemize}
    \item 元素为 $E_{\alpha}$ 的集合记为 $\{E_{\alpha}\}$,可以称其为集合群 collection of sets 或是类 family of sets。
    \item $S$ 代表所有 $E_{\alpha}$ 集合的并集;
    \item $P$ 代表所有 $E_{\alpha}$ 集合的交集。
  \end{itemize}
\end{anote}

\refstepcounter{poma}
\begin{remarks}
  Many properties of unions and intersections are quite similar to those of sums and products; in fact,
  the words sum and product were sometimes used in this connection, and the symbols $\Sigma$ and $\Pi$
  were written in place of $\bigcup$ and $\bigcap$.

  The commutative and associative laws are trivial:
  \begin{equation}
    A \cup B = B \cup A; \quad A \cap B = B \cap A.
  \end{equation}
  \begin{equation}
    (A \cup B) \cup C = A \cup (B \cup C); \quad (A \cap B) \cap C = A \cap (B \cap C).
  \end{equation}
  Thus the omission of parenthese in (3) and (6) is justified.

  The distributive law also holds:
  \begin{equation}
    A \cap (B \cup C) = (A \cap B) \cup (A \cap C).
  \end{equation}
  To prove this, let the left and right members of (10) be denoted by $E$ and $F$, respectively.

  Suppose $x \in E$. Then $x \in A$ and $x \in B \cup C$, that is, $x \in B$ or $x \in C$ (possibly both).
  Hence $x \in A \cap B$ or $x \in A \cap C$, so that $x \in F$. Thus $E \subset F$.

  Next, suppose $x \in F$. Then $x \in A \cap B$ or $x \in A \cap C$. That is, $x \in A$, and
  $x \in B \cup C$. Hence $x \in A \cap (B \cup C)$, so that $F \subset E$.

  It follows that $E = F$.

  We list a few more relations which are easily verified:
  \begin{equation}
    A \subset A \cup B,
  \end{equation}
  \begin{equation}
    A \cap B \subset A.
  \end{equation}
  If 0 denotes the empty set, then
  \begin{equation}
    A \cup 0 = A, \quad A \cap 0 = 0.
  \end{equation}
  If $A \subset B$, then
  \begin{equation}
    A \cup B = B, \quad A \cap B = A.
  \end{equation}
\end{remarks}

\begin{theorem}
  Let $\{E_n\},\ n=1,2,3,\dots$, be a sequence of countable sets, and put
  \begin{equation}
    S = \bigcup \limits_{n=1}^{\infty} E_n.
  \end{equation}
  Then $S$ is countable.
\end{theorem}

\begin{proof}
  Let every set $E_n$ be arranged in a sequence $\{x_{nk}\},\ k=1,2,3,\dots$, and consider the infinite array
  \begin{equation}
    \begin{tikzpicture}[baseline=(current bounding box.center)]
      \matrix[
        matrix of math nodes,
        nodes in empty cells,
      ](m) {
        x_{11} & x_{12} & x_{13} & x_{14} & \dots \\
        x_{21} & x_{22} & x_{23} & x_{24} & \dots \\
        x_{31} & x_{32} & x_{33} & x_{34} & \dots \\
        x_{41} & x_{42} & x_{43} & x_{44} & \dots \\
        \dots  & \dots  & \dots  & \dots  & \dots \\
      };
      \draw [->] (m-1-1.-150) -- (m-1-1.30);
      \draw [->] (m-2-1.-150) -- (m-1-2.30);
      \draw [->] (m-3-1.-150) -- (m-1-3.30);
      \draw [->] (m-4-1.-150) -- (m-1-4.30);
    \end{tikzpicture}
  \end{equation}

  in which the elements of $E_n$ form the $n$th row. The array contains all elements of $S$. As indicated by
  the arrows, these elements can be arranged in a sequence
  \begin{equation}
    x_11;x_21,x_12;x_32,x_22,x_13;x_41,x_32,x_23,x_14;\dots
  \end{equation}
  If any two of the sets $E_n$ have elements in common, these will appear more than once in (17). Hnce there is
  a subset $T$ of the set of all positive integers such that $S \sim T$, which shows that $S$ is at most coutable
  (Theorem 2.8). Since $E_1 \subset S$, and $E_1$ is infinite, $S$ is infinite, and thus countable.
\end{proof}

\begin{corollary}
  Suppose $A$ is at most coutable, and, for every $\alpha \in A$, $B_{\alpha}$ is at most countable. Put
  \[T = \bigcup\limits_{\aleph \in A} B_{\alpha}\ .\]
  Then $T$ is at most countable.
\end{corollary}

$T$ 相当于 (15) 的子集。

\begin{theorem}
  Let $A$ be a countable set, and let $B_n$ be the set of all $n$-tuples $(a_1,\dots,a_n)$, where
  $a_k \in A \ (k=1,\dots,n)$, and the elements $a_1,\dots,a_n$ need not be distinct. Then $B_n$ is coutable.
\end{theorem}

\begin{proof}
  $B_1$ 可数是显而易见的,因为 $B_1=A$。假设 $B_{n-1}$ 是可数的($n=2,3,4,\dots$)。$B_n$ 的元素形式是
  \begin{equation}
    (b,a) \quad (b \in B_{n-1}, \alpha \in A).
  \end{equation}
  对于每个固定的 $b$,成对集合(set of pairs)$b, a$ 等同于 $A$,即是可数的。因此 $B_n$ 是若干可数集合的并集构成的可数集合。根据 Theorem 2.12,
  $B_n$ 是可数的。
\end{proof}

\begin{corollary}
  The set of all rational numbers is countable.
\end{corollary}

\begin{proof}
  我们应用 Theorem 2.13 同时 $n=2$,所有有理数 $r$ 都可以表示为 $b/a$,其中 $a$ 与 $b$ 都是整数。那么成对集合 $(a, b)$ 就是
  分数 $b/a$ 的集合,即是可数的。
\end{proof}

实际上,所有代数集合都是可数的。

然而并不是所有的无限集合是可数的,详见下个定理。

\begin{theorem}
  Let $A$ be the set of all sequences whose elements are the digits 0 and 1. This set $A$ is uncoutable.
\end{theorem}

$A$ 集合中的元素数列类似于 $1, 0, 0, 1, 0, 1, 1, 1, \dots$。

\begin{proof}
  令 $E$ 为集合 $A$ 中的一个可数子集,且令 $E$ 由数列 $s_1, s_2, s_3, \dots$ 构成。再构建一个满足以下条件的数列 $s$。
  如果在 $s_n$ 中的第 $n$ 个小数是 1,令 $s$ 的第 $n$ 个小数为 0,以此类推。那么数列 $s$ 至少有一处是有别于所有 $E$ 中的成员;
  因此 $s \notin E$。但是陷入 $s \in A$,因此 $E$ 是 $A$ 的一个合理子集。

  我们证明了所有 $A$ 集合的可数子集是合理的子集。对于 $A$ 是不可数的也同理(否则 $A$ 将会是 $A$ 合理的子集,这是荒谬的)。
\end{proof}

\subsection*{Metric Spaces}

\begin{definition}
  A set $X$, whose elements we shall call \textit{points}, is said to be a \textit{metirc space} if with any two points
  $p$ and $q$ of $X$ there is associated a real number $d(p,q)$, called the \textit{distance} from $p$ to $q$, such that
  \begin{enumerate}[label=(\alph*)]
    \item $d(p,q)>0$ if $p \ne q; \ d(p, p) = 0$;
    \item $d(p,q) = d(q,p)$;
    \item $d(p,q) \le d(p,r) + d(r,q)$, for any $r \in X$.
  \end{enumerate}
\end{definition}

任何拥有上述三个性质的函数都被称为\textit{距离函数 distance function},或者\textit{度规 metric}。

\begin{examples}
  The most important examples of metric spaces, from our standpoint, are the euclidean spaces $R^k$, especially $R^1$
  (the real line) and $R^2$ (the complex plane); the distance in $R^k$ is defined by
  \begin{equation}
    d(\mathbf{x}, \mathbf{y}) = |\mathbf{x} - \mathbf{y}| \quad (\mathbf{x},\mathbf{y} \in R^k).
  \end{equation}
\end{examples}

\begin{anote}
  \begin{itemize}
    \item 欧几里得空间的距离概念抽象化后,即度量空间(Metric Space)。
    \item (a) 正定性,(b) 对称性,(c) 三角不等式。
  \end{itemize}
\end{anote}

\begin{definition}
  By the \textit{segment} $(a, b)$ we mean the set of all real numbers $x$ such that $a < x < b$.

  By the \textit{interval} $[a, b]$ we mean the set of all real numbers $x$ such that $a \le x \le b$.

  Occasionally we shall also encounter \say{half-open intervals} $[a, b)$ and $(a, b]$; the first consists of all $x$
  such that $a \le x < b$, the second of all $x$ such that $a < x \le b$.

  If $a_i < b_i$ for $i=1,\dots,k$, the set of all points $\mathbf{x} = (x_1,\dots,x_k)$ in $R^k$ whose coordinates
  satisfy the inequalities $a_i \le x_i \le b_i \ (1 \le i \le k)$ is called a \textit{k-cell}. Thus a 1-cell is an
  interval, a 2-cell is a rectangle, etc.

  If $\mathbf{x} \in R^k$ and $r>0$, the \textit{open (or closed) ball} $B$ with center at $\mathbf{x}$ and radius $r$
  is defined to be the set of all $\mathbf{y} \in R^k$ such that $|\mathbf{y}-\mathbf{x}|<r$ (or
  $|\mathbf{y}-\mathbf{x}| \le r$).

  We call a set $E \subset R^k$ \textit{convex} if
  \[\lambda\mathbf{x}+(1-\lambda)\mathbf{y} \in E\]
  whenever $\mathbf{x} \in E$, $\mathbf{y} \in E$, and $0<\lambda<1$.
\end{definition}

\newpage % 为下方图片预留空间
\begin{anote}
  \begin{itemize}
    \item Segment 开区间 $(a,b)$
    \item Interval 闭区间 $[a,b]$
    \item $k$-cell k-方格
    \item $R^k$ 空间定义\textbf{开/闭球(open/closed ball)}
    \item ball 和 $k$-cell 都是凸的(\textbf{convex})
  \end{itemize}
\end{anote}

\begin{figure}[h]
  \centering
  \includegraphics[width=0.3\textwidth]{\subfix{../images/k-cell.png}}\par
  k-cell
\end{figure}

\begin{definition}
  Let $X$ be a metirc space. All points and sets mentioned below are understood to be elements and subsets of $X$.
  \begin{enumerate}[label=(\alph*)]
    \item A \textit{neighborhood} of $p$ is a set $N_r(p)$ consisting of all $q$ such that $d(p,q)<r$, for some $r>0$.
          The number $r$ is called the \textit{radius} of $N_r(p)$.
    \item A point $p$ is a \textit{limit point} of the set $E$ if \textit{every} neighborhood of $p$ contains a point
          $q \ne p$ such that $q \in E$.
    \item If $p \in E$ and $p$ is not a limit point of $E$, then $p$ is called an \textit{isolated point} of $E$.
    \item $E$ is \textit{closed} if every limit point of $E$ is a point of $E$.
    \item A point $p$ is an \textit{interior} point of $E$ if there is a neighborhood $N$ of $p$ such that $N \subset E$.
    \item $E$ is \textit{open} if every point of $E$ is an interior point of $E$.
    \item The \textit{complement} of E (denoted by $E^c$) is the set of all points $p \in X$ such that $p \notin E$.
    \item $E$ is \textit{perfect} if $E$ is closed and if every point of $E$ is a limit point of $E$.
    \item $E$ is \textit{bounded} if there is a real number $M$ and a point $q \in X$ such that $d(p,q)<M$ for all
          $p \in E$.
    \item $E$ is \textit{dense} in $X$ if every point of $X$ is a limit point of $E$, or a point of $E$ (or both).
  \end{enumerate}
\end{definition}

注意 $R^1$ 的邻域为线段,而 $R^2$ 的邻域是圆圈的内点们。

\begin{anote}
  \begin{itemize}
    \item 邻域 neighborhood:到 $p$ 点距离小于 $r$ 的集合($r>0$);
    \item 极限点 limit point:所有邻域存在一个与 $p$ 不同的点,无论半径多小(例如一维空间的开闭区间 $(a,b],\ a<b, \ a,b \in R$
          的 $a,b$ 点皆为极限点,$a$ 虽然不属于 $(a,b]$,但是其右侧总是会有一个点 $p$ 使得 $p-a>0$,$b$ 同理。二维空间集合的
          边间点亦是如此,以此类推所有的边界点皆为极限点);
    \item 孤立点 isolated point:例如 $S = \{0\} \cup [1,2]$ 的 0 点为孤立点;
    \item 闭 closed:如果所有极限点都属于 $E$,那么 $E$ 为闭(例如一维空间 $S=[a,b],\ a<b, \ a,b \in R$,$S$ 为 closed。
          特例:在孤立点构成的度量空间中,任何子集都是即开又闭);
    \item 内点 interior point:用一维空间解释就是 $S=[a,b],\ a<b,\ a,b \in R$ 且 $p \in (a,b)$,那么 $p$ 为 $S$ 的内点;
    \item 开 open:如果 $E$ 中任意一点都是内点,$E$ 为开;
    \item 补集 complement:$\forall p \in X$ 且 $\forall p \notin E$ 那么 $X$ 为 $E$ 补集;
    \item 完全 perfect:一个闭集中每一个点都是它的极限点,那么该集合为完全;
    \item 有界的 bounded:如果集合中任意一点都在某个 $r$ 为实数的邻域内,那么该集合为有界的;
    \item 稠密 dense:$X$ 中任意一点都是 $E$ 的一个极限点或者 $E$ 中的一点(例如有理数在实数上稠密,Theorem 1.20 (b) 中证明了)。
  \end{itemize}
\end{anote}

\begin{theorem}
  Every neighborhood is an open set.
\end{theorem}

\begin{proof}
  考虑一个邻域 $E = N_r(p)$,并令 $q$ 为 $E$ 的任意一点。那么则有一个正实数 $h$ 满足
  \[d(p,q)=r-h.\]
  对于所有满足 $d(q,s)<h$ 的点 $s$,有
  \[d(p,s) \le d(p,q) + d(q,s) < r - h + h = r,\]
  使得 $s \in E$。因此 $q$ 是 $E$ 的一个内点。
\end{proof}

\anote 所有邻域都是开的。

\begin{theorem}
  If $p$ is a limit point of a set $E$, then every neighborhood of $p$ contains infinitely many points of $E$.
\end{theorem}

\begin{proof}
  假设 $p$ 有一个邻域 $N$,其仅包含了有限点的 $E$。令 $q_1,\dots,q_n$ 为 $N \cap E$ 的这些点,它们有别与点 $p$,且令
  \[r = \min\limits_{1 \le m \le n} d(p,q_m)\]
  (我们使用这个记号来表示最小的 $d(p,q_1),\dots,d(p,q_n)$)。一个有限集的最小值很明显是正数,因此 $r>0$。

  邻域 $N_r(p)$ 包含了除了点 $q$ 的 $E$ 即 $q \ne p$,使得 $p$ 不是 $E$ 的一个极限点。这与定理自身相悖。
\end{proof}

\anote 简单来说,假设 $p$ 有一个邻域是有限集合,那么就一定会有 $r$ 满足 $\min\limits_{1 \le m \le n} d(p,q_m)$,也就是
有最小距离存在;然而这与 Definition 2.18 (b) 相悖,即与\say{无论半径多小}的定义相悖,因此 $p$ 不是一个极限点。因此,
如果 $p$ 是一个极限点,那么它的任何邻域都有无限个点。

\begin{corollary}
  A finite point set has no limit points.
\end{corollary}

\begin{examples}
  Let us consider the following subsets of $R^2$:
  \begin{enumerate}[label=(\alph*)]
    \item The set of all complex z such that $|z| < 1$.
    \item The set of all complex z such that $|z| \le 1$.
    \item A nonempty finite set.
    \item The set of all integers.
    \item The set consisting of the numbers $1/n \ (n=1,2,3,\dots)$. Let us note that this set $E$ has a limit point
          (namely, $z=0$) but that no point of $E$ is a limit point of $E$; we wish to stress the difference between having
          a limit point and containing one.
    \item The set of all complex numbers (that is, $R^2$).
    \item The segment $(a,b)$.
  \end{enumerate}

  Let us note that (d), (e), (g) can be regarded also as subsets of $R^1$. Some properties of these sets are tabulated
  below:

  \begin{center}
    \begin{tikzpicture}[baseline=(current bounding box.center)]
      \matrix[
        matrix of math nodes,
        nodes in empty cells,
      ]
      {
                            & Closed & Open & Perfect & Bounded \\
        (a) \quad\quad\quad & No     & Yes  & No      & Yes     \\
        (b) \quad\quad\quad & Yes    & No   & Yes     & Yes     \\
        (c) \quad\quad\quad & Yes    & No   & No      & Yes     \\
        (d) \quad\quad\quad & Yes    & No   & No      & No      \\
        (e) \quad\quad\quad & No     & No   & No      & Yes     \\
        (f) \quad\quad\quad & Yes    & Yes  & Yes     & No      \\
        (g) \quad\quad\quad & No     &      & No      & Yes     \\
      };
    \end{tikzpicture}
  \end{center}

  In (g), we left the second entry blank. The reason is that the segment $(a,b)$ is not open if we regard it as a subset
  of $R^2$, but it is an open subset of $R^1$.
\end{examples}

\begin{theorem}
  Let $\{E_{\alpha}\}$ be a (finite or infinite) colllection of sets $E_{\alpha}$. Then
  \begin{equation}
    \biggl(\bigcup\limits_{\alpha} E_{\alpha}\biggr)^c = \bigcap\limits_{\alpha} (E_{\alpha}^c).
  \end{equation}
\end{theorem}

\begin{proof}
  令 $A$ 与 $B$ 分别为 (20) 的左右成员。如果 $x \in A$,那么 $x \notin \bigcup_{\alpha}E_{\alpha}$,因此对于任何 $\alpha$
  而言 $x \notin E_{\alpha}$,因此对于所有 $\alpha$ 而言 $x \in E_{\alpha}^c$,使得 $x \in \bigcap E_{\alpha}^c$。
  所以 $A \subset B$。

  相反的,如果 $x \in B$,那么对于所有 $\alpha$ 而言 $x \in E_{\alpha}^c$,因此对于任何 $\alpha$ 而言 $x \notin E$,
  因此 $x \notin \bigcup_{\alpha} E_{\alpha}$,使得 $x \in (\cup_{\alpha} E_{\alpha})$。所以 $B \subset A$。

  综上所述 $A = B$。
\end{proof}

\begin{theorem}
  A set $E$ is open if and only if its complement is closed.
\end{theorem}

\begin{proof}
  首先,假设 $E^c$ 是闭的。选 $x \in E$,且 $x \notin E^c$,且 $x$ 不是 $E^c$ 的一个极限点。因此 $x$ 存在一个邻域 $N$ 使得
  $E^c \cap N$ 是空集,也就是说 $N \subset E$。因此 $x$ 是 $E$ 的一个内点,且 $E$ 是开的。

  接着,假设 $E$ 是开的。令 $x$ 为 $E^c$ 的一个极限点。那么 $x$ 的所有邻域包含 $E^c$ 的一个点,使得 $x$ 不是 $E$ 的一个内点。
  由于 $E$ 是开的,这意味着 $x \in E^c$。即遵循了 $E^c$ 是闭的。
\end{proof}

\begin{corollary}
  A set $F$ is closed if and only if its complement is open.
\end{corollary}

\begin{theorem}\mbox{}
  \begin{enumerate}[label=(\alph*)]
    \item For any collection $\{G_{\alpha}\}$ of open sets, $\cup_{\alpha}G_{\alpha}$ is open.
    \item For any collection $\{F_{\alpha}\}$ of closed sets, $\cap_{\alpha}F_{\alpha}$ is closed.
    \item For any finite collection $G_1,\dots,G_n$ of open sets, $\cap_{i=1}^n G_i$ is open.
    \item For any finite collection $F_1,\dots,F_n$ of closed sets, $\cup_{i=1}^n F_i$ is closed.
  \end{enumerate}
\end{theorem}

\begin{proof}
  令 $G=\cup_{\alpha}G_{\alpha}$。如果 $x \in G$ 那么对一些 $\alpha$ 而言,有 $x \in G_{\alpha}$。因为 $x$ 是 $G_{\alpha}$
  的一个内点,同时也是 $G$ 的一个内点,所以 $G$ 是开的。即证明了 (a)。

  根据 Theorem 2.22,
  \begin{equation}
    \biggl(\bigcap\limits_{\alpha}F_{\alpha}\biggr)^c = \bigcup\limits_{\alpha}(F_{\alpha}^c)
  \end{equation}
  且 $F_{\alpha}^c$ 是开的,根据 Theorem 2.23。因此 (a) 意为 (21) 是开的,因此 $\cap_{\alpha}F_{\alpha}$ 是闭的。

  接着令 $H=\cap_{i=1}^n G_i$。对于任何 $x \in H$,存在 $x$ 半径 $r_i$ 的邻域 $N_i$,使得 $N_i \subset G_i \ (i=1,\dots,n)$。
  令
  \[r=\min(r_1,\dots,r_n)\]
  且令 $N$ 为 $x$ 半径 $r$ 的邻域。那么对于 $i=1,\dots,n$ 而言 $N \subset G_i$ 使得 $N \subset H$,所以 $H$ 是开的。

  通过获取补集,(d) 遵循 (c):
  \[\biggl(\bigcup\limits_{i=1}^n F_i\biggr)^c = \bigcap\limits_{i=1}^n(F_i^c)\]
\end{proof}

\begin{anote}
  对于 (a) 而言,利用了开集的定义 Definition 2.18 (f),也就是说 $x$ 可以是 $G$ 的任意一点,且因为 $x \in G_{\alpha}$ 都是内点,
  满足开集定义。

  对于 (b) 而言,将 Theorem 2.23 应用在 Theorem 2.22 即可得出结论。

  对于 (c) 而言,$H$ 是有限个数开集的交集,那么对于每个构成 $H$ 的 $G_i$ 而言都有一个 $N_i$ 作为 $x$ 的邻域,那么在有限个集合内
  肯定可以找到最小的邻域(最小半径 $r$),又因为该最小的邻域为所有 $G_i$ 的子集,即 $N \subset G_i, \ i=1,\dots,n$,因此有
  $N \subset H$。最后因为对于任意 $x \in H$ 而言,上述论证的 $N_x \subset H$ 都成立,即符合开集定义。

  对于 (d) 而言,再次将 Theorem 2.23 应用在 (c) 上 即可得出结论。

  简言之:
  \begin{itemize}
    \item 任意多的开集的并集仍然是开的;
    \item 任意多的闭集的交集仍然是闭的;
    \item 有限个的开集的交集仍然是开的;
    \item 有限个的闭集的并集仍然是闭的。
  \end{itemize}
\end{anote}

\refstepcounter{poma}
\begin{definition}
  If $X$ is a metric space, if $E \subset X$, and if $E$'s denotes the set of all limit points of $E$ in $X$, then the
  \textit{closure} of $E$ is the set $\overline{E} = E \cup E'$.
\end{definition}

\anote $E'$ 为所有极限点的集,那么 $\overline{E} = E \cup E'$ 为闭包(例如一维空间 $(a, b), \ a<b, \ a,b \in R$ 有极限点
$a, \ b$,那么 $\overline{E} = (a,b) \cup {a} \cup {b} = [a,b]$)。

\begin{theorem}
  If $X$ is a metric space and $E \subset X$, then
  \begin{enumerate}[label=(\alph*)]
    \item $\overline{E}$ is closed,
    \item $E = \overline{E}$ if and only if $E$ is closed,
    \item $\overline{E} \subset F$ for every closed set $F \subset X$ such that $E \subset F$.
  \end{enumerate}
  By (a) and (c), $\overline{E}$ is the \textit{smallest} closed subset of $X$ that contains $E$.
\end{theorem}

\begin{proof}
  \begin{enumerate}[label=(\alph*)]
    \item 如果 $p \in X$ 且 $p \notin \overline{E}$,那么 $p$ 既不是 $E$ 的点也不是 $E$ 的极限点。因此 $p$ 有一个邻域不与
          $E$ 相交。那么 $\overline{E}$ 的补集是开的。因此 $\overline{E}$ 是闭的。
    \item 如果 $E = \overline{E}$,那么 (a) 意味着 $E$ 是闭的。如果 $E$ 是开的,那么 $E' \subset E$(根据 Definitions 2.18 (d)
          与 2.26),因此 $\overline{E} = E$。
    \item 如果 $F$ 是闭的且 $F \supset E$,那么 $F \supset F'$,因此 $F \supset E'$。所以 $F \supset \overline{E}$。
  \end{enumerate}
\end{proof}

\anote
对于 (c) 而言,如果 $E \subset F$,那么 $E$ 的极限点要么属于 $F$ 的内点,要么属于 $F$ 的极限点;而根据闭集的定义,
闭集中所有的极限点都属于闭集,所以 $E \cup E' \subset F$,即 $\overline{E} \subset F$。

\begin{theorem}
  Let $E$ be a nonempty set of real numbers which is bounded above. Let $y = \sup E$. Then $y \in \overline{E}$.
  Hence $y \in E$ if $E$ is closed.
\end{theorem}

\begin{proof}
  如果 $y \in E$ 那么 $y \in \overline{E}$。假设 $y \notin E$。那么对于所有的 $h>0$ 存在一个点 $x \in E$ 使得 $y-h<x<y$,
  不然的话 $y-h$ 会是 $E$ 的一个上界。因此 $y$ 是 $E$ 的一个极限点,所以 $y \in \overline{E}$。
\end{proof}

\anote 如果 $E$ 是闭集,而 $y \notin E$ 时,那么总会有 $h>0$ 使得 $y-h$ < $y$ 为 $E$ 的上界,那么与 $y = \sup E$ 相悖。

\begin{remark}
  Suppose $E \subset Y \subset X$, where $X$ is a metric space. To say that $E$ is an open subset of $X$ means that to
  each point $p \in E$ there is associated a positive number $r$ such that the conditions $d(p,q)<r,\ q \in X$ imply
  that $q \in E$. But we have already observed (Examples 2.16) that $Y$ is also a metric space, so that our definitions
  may equally well be made within $Y$. To be quite explicit, let us say that $E$ is \textit{open relative} to $Y$ if to
  each $p \in E$ there is associated an $r>0$ such that $q \in E$ whenever $d(p,q)<r$ and $q \in Y$. Examples 2.21(g)
  showed that a set may be open relative to $Y$ without being an open subset of $X$. However, there is a simple relation
  between these concepts, which we now state.
\end{remark}

\anote
根据 Examples 2.16 与 2.21 (g) 引入相对开集这个概念,在下面的定理中证明相对开集 $E$ 与 $Y$ 和 $X$ 的关系。

\begin{theorem}
  Suppose $Y \subset X$. A subset $E$ of $Y$ is open relative to $Y$ if and only if $E = Y \cap G$ for some open
  subset $G$ of $X$.
\end{theorem}

\begin{proof}
  假设 $E$ 是相对 $Y$ 是开。每个 $p \in E$ 都存在一个正数 $r_p$ 满足 $d(p,q)<r_p,\ q \in Y$ 意味着 $q \in E$。
  令 $V_p$ 为所有 $q \in X$ 的集,使得 $d(p,q)<r_p$,并且定义
  \[G = \bigcup\limits_{p \in E} V_p\]
  那么根据 Theorems 2.19 与 2.24 可知 $G$ 是 $X$ 的一个子开集。

  因为对于所有 $p \in E$ 而言 $p \in V_p$,即 $E \subset G \cap Y$。

  对于 $V_p$ 而言,对于每个 $p \in E$ 有 $V_p \cap Y \subset E$,使得 $G \cap Y \subset E$。因此 $E = G \cap Y$,
  这样该定理的一半被证明了。

  相反的,如果 $G$ 在 $X$ 中是开的,且对于每个 $x \in E$ 而言 $E = G \cap Y$ 有一个邻域 $V_p \subset G$。那么
  $V_p \cap Y \subset E$,使得 $E$ 相对 $Y$ 是开的。
\end{proof}

\begin{anote}
  证明的一开始\say{假设 $E$ 是相对 $Y$ 是开}利用了 Remark 2.29 的结论(而 Remark 2.29 的前半结论是由 Examples 2.16 所得出的):
  \say{to each point $p \in E$ there is associated a positive number $r$ such that the conditions $d(p,q)<r,\ q \in X$
    imply that $q \in E$}。接下来,Theorem 2.19 说的是\say{所有邻域都是开集},而证明中 Theorem 2.24 (a) 说的是若干开集的并集
  也是开集。简言之 $p$ 在 $X$ 上有邻域,且对所有 $p \in E$ 有效,那么 $G=\bigcup\limits_{p \in E} V_p$ 在 $X$ 中是开的。

  又因为对于所有的 $p \in E$ 都有 $p \in Vp$,那么就有 $E \subset G$;同时 $E \subset Y$,所以有 $E \subset G \cap Y$。

  而证明中设定的 $V_p$ 本身就是 $p$ 点在 $Y$ 上的邻域,即 $V_p \cap Y \subset E$;这又对于所有 $p \in E$ 生效,即
  $G \cap Y \subset E$。结合上面的结论有 $E \subset G \cap Y \subset E$,即 $E = G \cap Y$。

  而 $E$ 相对 $Y$ 为开的证明就简单多了,因为 $p \in E$ 在 $G$ 上的邻域有 $V_p \subset G$,而 $G \cap Y=E$ 且 $V_p \subset G$
  可知 $V_p \cap Y \subset E$,即 $p \in E$ 在 $Y$ 上的邻域满足 Theorem 2.19,再结合 Theorem 2.24 (a),可得结论。

  性质:集合的\textbf{开}性质是相对的。相对 $Y$ 的开集 $E$,对于更大的 $X$ 来说不一定是开的。要证明对于更大的 $X$ 是开,就必须要
  引入 $X$ 的开子集 $G$。
\end{anote}

\end{document}
